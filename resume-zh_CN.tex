% !TEX TS-program = xelatex
% !TEX encoding = UTF-8 Unicode
% !Mode:: "TeX:UTF-8"

\documentclass{resume}
\usepackage{zh_CN-Adobefonts_external} % Simplified Chinese Support using external fonts (./fonts/zh_CN-Adobe/)
%\usepackage{zh_CN-Adobefonts_internal} % Simplified Chinese Support using system fonts
\usepackage{linespacing_fix} % disable extra space before next section
\usepackage{cite}
\usepackage[colorlinks=true,urlcolor = cyan]{hyperref}

\begin{document}
\pagenumbering{gobble} % suppress displaying page number

\name{池克俭}

% {E-mail}{mobilephone}{homepage}
% be careful of _ in emaill address
\contactInfo{(+86) 180-2797-9397}{kejian2531693734@gmail.com}{软件开发工程师}{\href{https://github.com/Chiichen}{Github@Chiichen}}
% {E-mail}{mobilephone}
% keep the last empty braces!
%\contactInfo{xxx@yuanbin.me}{(+86) 131-221-87xxx}{}

% \section{\faGraduationCap\ 教育背景}
\section{教育背景}
\datedsubsection{\textbf{华南理工大学},软件工程,\textit{在读本科生}}{2021.9 - 2025.6}
\ 获校级奖学金, 任IBM校园大使, 任IBM学生俱乐部、微软俱乐部主席团成员

% \section{\faCogs\ IT 技能}
\section{技术能力}
% increase linespacing [parsep=0.5ex]
\begin{itemize}[parsep=0.2ex]
  \item \textbf{编程语言}: Rust, C++, C, Go, Java, Python
  \item \textbf{基本技能}: Linux, Mysql/MongoDB/Postgresql, Git/Github/Github Workflow, Makefile/Cmake/Xmake, GDB, Qemu
  \item \textbf{开发框架与技术栈}: QT, Spring, SpringCloud, Mybatis(plus), Gin, Gorm, gRPC
\end{itemize}

% \end{itemize}

\section{主要项目经历}
\datedsubsection{\textbf{EasyCode企业级低代码开发平台}, 项目负责人,后端开发工程师}{2023.8 - 2023.11}
\begin{itemize}
%   \item 飞猪北京前端团队全面负责各交通线的票务(机票/火车票/汽车票) web 应用与事业群基础架构研发
  \item \textbf{负责项目整体功能与架构设计、协调前后端开发,负责安排项目成员的周期工作。}提出了“中”代码平台的设计方案,相比于市面上的“表单驱动型”或者“流程驱动型”低代码平台,大大提升了平台的功能上限。基于“企业级”的项目要求,设计了基于K8S的分布式架构,对于不同的应用实例实现隔离。
  \item \textbf{负责后端代码的代码审查以及基于Github Action的CI/CD流程。}通过Github基于Pull Request的开发流程对提交的代码进行审查,大大降低了引入腐败代码的概率。并且通过Github Action,实现代码自动测试。通过编写脚本,使得代码可以通过一行命令启动容器,大大降低了部署门槛和测试门槛。
  \item \textbf{在后端总计四个微服务模块中独立负责两个模块的开发,主要负责一个模块的开发。}独立负责用户鉴权服务以及应用后台的开发。在前者开发中使用了Gin + redis缓存 + JWT 验证的认证方案,使用 postgres 作为主要数据库。而后者的开发中实现了对postgres、mongodb等数据库的支持,使得前端可以通过统一的接口来访问异构的数据库。主要负责对网站后端的开发,主要使用了Gin作为Web框架以及MongoDB作为主力数据库,主要用于存储前端传入的应用布局信息,再通过接口暴露给前端来实现低代码应用的设计与部署
\end{itemize}

\datedsubsection{\textbf{DragonOS龙操作系统},系统内核工程师,主要维护者}{2023.7 - 至今}
\begin{itemize}
  \item \textbf{负责内核模块开发。}深度参与了包括进程管理、进程间通信、设备驱动及管理在内的多个内核模块的开发,熟练使用gdb进行内核调试,在开发过程中对内核存在的诸如死锁问题、中断重入问题进行处理。
  \item \textbf{用户程序移植与测试。}基于MUSL工具链进行常见用户程序的重编译与移植,并尝试解决移植过程中可能出现的诸如缺少系统调用/预期行为不一致等问题。使用MUSL工具链编写用户程序,测试内核模块的工作状态或者进行系统性能测试
  \item \textbf{负责代码审查与开发任务制定。}作为主要维护者,通过Github基于Pull Request的开发流程进行代码审查并进行代码风格管理。与其它主要开发者一同进行项目阶段性开发任务的制定,在完善了进程管理、文件系统、内存管理等底层模块后,转向由功能驱动的开发模式,近期项目已完成对reids-server的移植。
  \item \textbf{参与DragonOS Community开源社区建设。}负责项目成员招募,并通过发布项目相关视频等方式参与社区自媒体账号建设。
\end{itemize}



% \begin{onehalfspacing}
% \end{onehalfspacing}

% \datedsubsection{\textbf{DID-ACTE} 荷兰莱顿}{2015年3月 - 2015年6月}
% \role{本科毕业设计}{LIACS 交换生}
% 利用结巴分词对中国古文进行分词与词性标注,用已有领域知识训练形成 classifier 并对结果进行调优
% \begin{onehalfspacing}
% \begin{itemize}
%   \item 利用结巴分词对中国古文进行分词与词性标注
%   \item 利用已有领域知识训练形成 classifier, 并用分词结果进行测试反馈
%   \item 尝试不同规则,对 classifier 进行调优
% \end{itemize}
% \end{onehalfspacing}



% \section{\faHeartO\ 项目/作品摘要}
% \section{项目/作品摘要}
% \datedline{\textit{An Integrated Version of Security Monitor Vis System}, https://hijiangtao.github.io/ss-vis-component/ }{}
% \datedline{\textit{Dark-Tech}, https://github.com/hijiangtao/dark-tech/ }{}
% \datedline{\textit{融合社交网络数据挖掘的电视节目可视化分析系统}, https://hijiangtao.github.io/variety-show-hot-spot-vis/}{}
% \datedline{\textit{LeetCodeOJ Solutions}, https://github.com/hijiangtao/LeetCodeOJ}{}
% \datedline{\textit{Info-Vis}, https://github.com/ISCAS-VIS/infovis-ucas}{}


% \section{\faInfo\ 社会实践/其他}
\section{其它项目}
% increase linespacing [parsep=0.5ex]
\begin{itemize}[parsep=0.2ex]
  \item \textbf{个人技术博客 \href{https://chiichen.github.io/}{ChiChen's Blog}}:https://chiichen.github.io/
  \item \textbf{\href{https://github.com/MaaAssistantArknights/MaaAssistantArknights}{MaaAssistantArknights}(Github 10.9K Star)}: C++实现的基于图像识别的明日方舟助手
  \item \textbf{\href{https://github.com/RccCommunity/rcc}{rcc 编译器}}:用Rust实现的编译器,尚处于demo阶段,有现代化的编译器错误处理
  \item \textbf{\href{https://github.com/Chiichen/sponge}{CS144}}:C++实现的TCP/IP协议栈
\end{itemize}

%% Reference
%\newpage
%\bibliographystyle{IEEETran}
%\bibliography{mycite}
\end{document}
